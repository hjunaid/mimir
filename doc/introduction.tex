\Mimir\ is a multi-paradigm information management index and repository which
can be used to index and search over text, annotations, semantic schemas
(ontologies), and semantic meta-data (instance data). It allows queries that
arbitrarily mix full-text, structural, linguistic and semantic queries and that
can scale to gigabytes of text.

A typical semantic annotation project deals with large quantities of data of
different kinds. \Mimir\ provides a framework for implementing indexing and
search functionality across all these data types, listed below in the order of
increasing information density:

{\bf Text}\\
All documents have a textual content. Support for full text search represents
the most basic indexing functionality and it is required in most (if not all)
cases.  Even when semantic annotation is used to abstract away from the
actual textual data, the original content still needs to be accessible so
that it can be used to provide textual query fragments in the case of more
complex conceptual queries.

\Mimir\ uses inverted indexes\footnote{{\em Inverted Indexes} are data
structures traditionally used in Information Retrieval to support indexing of
text.} for indexing the document content (including additional linguistic
information, such as part-of-speech or morphological roots), and for
associating instance of annotations with the position in the input text where
they occur. The inverted index implementation used by \Mimir\ is based on
MG4J\footnote{\url{http://mg4j.dsi.unimi.it/}}.

{\bf Annotations}\\
The first step in abstracting away from the plain text content is the
production of {\em annotations}. Annotations are meta-data associated to text
snippets in the documents. \Mimir's view of annotations is based on that of
GATE, with each annotation described by
\begin{itemize}
  \item the document it belongs to;
  \item the start and end offset of the referred text snippet;
  \item the annotation type;
  \item an arbitrary set of \verb!<!feature,value\verb!>! pairs.
\end{itemize}

An annotation index supports a more generic search paradigm. Depending on the
type of annotations available, the user can search across different dimensions.
If, for example, the documents are annotated with occurrences of {\tt Person,
Location, Organization} entities, then searches like {\tt \{Person\}, CEO of
\{Organization\}, based in \{Location\}} become possible.  Storage of
annotation data in \Mimir\ indexes is handled by plugins, \Mimir\ ships with
two storage plugins by default, one storing annotation data in a relational
database and the other in a Knowledge Base to support richer semantic querying.

ANNIC (ANNotations In
Context)\footnote{See \url{http://gate.ac.uk/userguide/chap:annic}.} is a tool
predating \Mimir\ that supports the indexing of annotations, and that has been
used to inform the design of \Mimir.

{\bf Knowledge Base Data}\\
Knowledge Base (KB) Data  consists of an ontology populated with instances. The
ontology represents the data schema and comprises a hierarchy of class types
and a hierarchy of properties that are applicable between instances of classes.
The instance data represents facts that are known to the systems and is
typically at least partially derived from semantic annotation over documents.
KB data is used to reach a higher level of abstraction over the information in
the documents which enables conceptual queries such as ``find all mentions of
{\tt Person}s who are employed by any organisation based in Yorkshire''.

A KB that is pre-populated with appropriate world knowledge can perform other
generalisations that are natural to humans users, such as being able to identify
Vienna as a valid answer to queries relating to Austria, Europe or the Western
Hemisphere.

As mentioned above, \Mimir\ can make use of a Knowledge Base to store
information relating to annotations. The links between annotations, the textual
data, and the knowledge base information are created by the inclusion into the
text indexes of a set specially-created URIs that are associated with
annotation data. Furthermore, URIs of entities from the Knowledge Base can be
stored as annotation features.

Knowledge bases are typically represented as a collection of triples that are
kept in highly-specialised and optimised triple stores, using standards such as
RDF or one the versions of OWL\footnote{See
\url{http://www.w3.org/RDF/} and \url{http://www.w3.org/TR/owl-features/}.}. The
implementation used by \Mimir\ is based on ORDI and
OWLIM\footnote{See
\url{http://www.ontotext.com/ordi/} and \url{http://www.ontotext.com/owlim/}.}.

\section{Core Concepts}

\Mimir{} provides indexing infrastrucutre for annotated
GATE\footnote{\url{http://gate.ac.uk}} documents. Users can start a \Mimir{}
server, submit documents to it for indexing, and execute queries against the set
of indexed documents. 

A \Mimir{} index is a composite of multiple sub-indexes, which are defined in
the {\em index template} that needs to be provided by the user when a new
\Mimir{} index is created (see Section~\ref{sec:indexing:templates} for
details).

{\bf Token Indexes} are sub-indexes that store the information associated with 
\verb!{Token}! annotations. These are provide a way to index the document
content. \Mimir{} does not directly index the document text. Instead it uses the
sequence of \verb!{Token}! annotation to construct a representation of the
document text. This provides more flexibility: if the user chooses to index the
{\tt string} feature of the tokens, that is equivalent to indexing the document
text. Alternatively, the user could chose to pre-process their document with the
GATE Morphological Analyser, and instead index the morphological roots of each
token. This normalises the representation of words (by eliminating inflections)
and allows different forms of the same word to be matched (e.g. {\em house} and
{\em houses}). This is similar to stemming/lemmatising, a process traditionally
employed in Information Retrieval, but it is more advanced and linguistically
sophisticated, and allows matching e.g. {\em be}, {\em was}, {\em are} with
each-other, which stemming would not be capable to.

Beside allowing the user to choose which token feature should be indexed,
\Mimir{} also allows multiple token features to be indexed in parallel
sub-indexes. The user can actually choose to index {\bf both} the token string
and morphological root. In that case, the feature mentioned first in the
{\em index template} becomes the default token feature. To search on any of the
other token features, queries need to specify which feature they want to target
(see Section~\ref{sec:string-query} for details).

{\bf Annotation Indexes} are the other type of \Mimir{} sub-index. They are used
to index information about annotations on the document. Which annotations should
be indexed is described in the {\em index template}.

Both token and annotation indexes can be configured to also use {\bf direct
indexes}. Direct can be used to perform searches for terms starting from
documents, for eaxmple finding the most frequently occurring word (or
annotation) in a set of documents. This functionality is only available from the
Java API and cannot be directly accessed by the system users via the web
interface. More details can be found in Section~\ref{sec:direct-indexes}.

\section{\Mimir{} Lifecycle}

In vesions prior to $5.0$, a \Mimir{} index would start its existence in {\em
indexing} mode, when it would accept new documents for indexing. When all the
documents had been indexed, the index would need to be {\em closed}, which would
switch its operation mode to {\em searching}, and the index would then be able
to answer queries. Once closed, and index could not accept any further documents
for indexing. Starting with version $5.0$, a Mímir index is continually
accepting documents to be indexed and can answer queries that address the
currently indexed document set. From being sent to \Mimir{} for indexing to
becoming avaialble for search, documents go through several stages, which we
describe next.

Documents submitted for indexing are initially accumulated in RAM, during
which time they are not available for being searched. A {\em sync-to-disk}
operation writes all the documents currently in RAM to disk, in the form of an
{\em index batch}, after which the documents can be searched. Sync-to-disk
operations happen automatically when too much document data has been accumulated
in RAM, or after a given time interval has passed since the last sync.
Alternatively, the user can also trigger a sync operation from index admin web
interface.

Every sync-to-disk operation causes a new index {\em batch} to be created.
All the batches are merged into a index cluster which is then used to serve
queries. If the number of clusters gets too large, it can harm efficiency or the
system can run into problems due to too large a number of files being open. To
avoid this, the index batches can be compacted into a single batch. \Mimir{}
indexes will automatically do that once the number of batches exceeds a certain
threshold (which can be modified via API calls).

In order to keep its consistency, a \Mimir{} index {\bf must} be closed in an
orderly fashion before the mimir server process is shut down. Shutting down the
\Mimir{} server (e.g. the {\tt mimir-cloud} web application) will automatically
close all currently open indexes. Users should never forcefully destroy the
\Mimir{} server process, as that would not allow the close operations to be
performed, which can lead to data loss, or it can corrupt existing indexes.
