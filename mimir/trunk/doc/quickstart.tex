This chapter is aimed at the impatient reader who wants a working system as
quickly as possible. The technical detail is deliberately kept at a minimum so,
while you will hopefully end up with something that works, you will not
necessarilly understand how it all fits together. For that, please read the
remainder of this guide.

We suggest you try this on a 64~bit operating system, as that is better suited
for running \Mimir{}. A 32~bit system would also work, but the maximum sizes for
the indexes wold be limited.

In order to build and run a \Mimir{} server you will need the following pieces
of software installed on your system:
\begin{description}
  \item[Java Development Kit] If you don't have one, you can download one from
  Oracle\footnote{\url{http://www.oracle.com/technetwork/java/javase/downloads/index.html}}.
  Make sure you get the JDK and not the Java Runtime Environment (JRE), as that
  would not be suitable. Once installed, make sure your \verb!JAVA_HOME!
  environment variable points to the location where the JDK was installed. 
  \item[Apache ANT] version 1.8.1 or later. You can download it from
  \url{http://ant.apache.org/}. Once installed, make sure your \verb!ANT_HOME!
  environment variable points to the top-level directory of your installation.
  \item[Grails] version 1.3.7. You can download this from
  \url{http://grails.org}. Once installed, make sure your \verb!GRAILS_HOME!
  environment variable points to the top-level directory of your installation. 
\end{description}

After all these prerequisites are installed, we can move to building a \Mimir{}
application. For the purposes of this demo, we will build the {\tt mimir-demo}
application. In a real deployment you may find the alternative application ({\tt
mimir-cloud}) more suitable.

To build {\tt mimir-demo} follow these steps:
\begin{enumerate}
  \item download \Mimir{}
  \item build the library
  \item generate the demo application
  \item build the demo application
\end{enumerate}


