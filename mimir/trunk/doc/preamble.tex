%%%%%%%%%%%%%%%%%%%%%%%%%%%%%%%%%%%%%%%%%%%%%%%%%%%%%%%%%%%%%%%%%%%%%%%%%%%%
%
% Valentin Tablan 13/02/2002
%
% $Id: preamble.tex,v 1.8 2004/07/14 13:51:44 valyt Exp $
%
% This file is designed to be included at the begining of any latex document 
% and handles things like pdf detection, tex4ht detection, notes command, 
% paragraph spacing and indenting.
%
% 28/July/2003: consolidated GATE utils and TAO main.tex stuff here (Hamish).
%
%%%%%%%%%%%%%%%%%%%%%%%%%%%%%%%%%%%%%%%%%%%%%%%%%%%%%%%%%%%%%%%%%%%%%%%%%%%%


%%%%%%%%%%%%%%%%%%%%%%%%%%%%%%%%%%%%%%%%%%%%%%%%%%%%%%%%%%%%%%%%%%%%%%%%%%%%
% PDF detection
%%%%%%%%%%%%%%%%%%%%%%%%%%%%%%%%%%%%%%%%%%%%%%%%%%%%%%%%%%%%%%%%%%%%%%%%%%%%

\usepackage{ifpdf}

%%%%%%%%%%%%%%%%%%%%%%%%%%%%%%%%%%%%%%%%%%%%%%%%%%%%%%%%%%%%%%%%%%%%%%%%%%%%
% tex4ht detection
%%%%%%%%%%%%%%%%%%%%%%%%%%%%%%%%%%%%%%%%%%%%%%%%%%%%%%%%%%%%%%%%%%%%%%%%%%%%

\newif\ifhtml
\ifx\ifHtml\undefined
  \htmlfalse          % we are not running tex4ht
\else
  \htmltrue           % we are running tex4ht
\fi


%%%%%%%%%%%%%%%%%%%%%%%%%%%%%%%%%%%%%%%%%%%%%%%%%%%%%%%%%%%%%%%%%%%%%%%%%%%%
% packages
%%%%%%%%%%%%%%%%%%%%%%%%%%%%%%%%%%%%%%%%%%%%%%%%%%%%%%%%%%%%%%%%%%%%%%%%%%%%

% note that hyperref modifies many of the latex commands so it needs to be 
% the last loaded package.

% general packages
\usepackage[english]{babel}
\usepackage{lscape}
\usepackage{longtable}
\usepackage{setspace}
\usepackage{color}
\usepackage{fancyhdr}
\usepackage{listings}
\usepackage{longtable}
\usepackage{txfonts}
\usepackage{upquote}
\usepackage[text={5in,8.5in}, hmarginratio=1:1, vmarginratio=1:1]{geometry}
% packages with mode detection 
\ifpdf
  %PDF specific stuff
  \usepackage[pdftex]{graphicx}
  % prefer .pdf graphics if available
  \DeclareGraphicsExtensions{.pdf,.png}
  \usepackage{thumbpdf}

  % this should be last!
  \usepackage[
    pdftex,
    pdfborder=0,
    pdftitle={},
    pdfauthor={},
    pdfsubject={},
    pdfkeywords={},
    pdfpagemode=None,
    colorlinks=false,
    urlcolor=black,
    anchorcolor=black,
    anchorcolor=black,
    filecolor=black,
    linkcolor=black,
    citecolor=black,
    hidelinks
    ]{hyperref}
\else
  \ifhtml
    % tex4ht specific stuff
    \usepackage{graphicx}
    \usepackage[colorlinks=false]{hyperref}
    % force .png extension on includegraphics
    \DeclareGraphicsExtensions{.png}
  \else
    % Latex specific stuff
    \usepackage[dvips]{graphicx}

    % this should be last!
    \usepackage[hypertex, colorlinks=false]{hyperref}
  \fi
\fi

%%%%%%%%%%%%%%%%%%%%%%%%%%%%%%%%%%%%%%%%%%%%%%%%%%%%%%%%%%%%%%%%%%%%%%%%%%%%
% HTML support commands
%%%%%%%%%%%%%%%%%%%%%%%%%%%%%%%%%%%%%%%%%%%%%%%%%%%%%%%%%%%%%%%%%%%%%%%%%%%%
% some utility definitions of HTML commands using the hyperref package 
% \htlink{target URL}{link text} - hypertext link
\newcommand{\htlink}[2]{\href{#1}{#2}}
\newcommand{\htlinkplain}[1]{\href{#1}{#1}}

% latex commands that only appear in the HTML version
\newcommand{\htlinkhtml}[2]{\href{#1}{#2}}
\newcommand{\htlinkplainhtml}[1]{\href{#1}{#1}}
\newcommand{\htlatexhtml}[1]{#1}

% \htcode{code} - raw HTML code
\newcommand{\htcode}[1]{}

% \htlatex{code} - latex only code
\newcommand{\htlatex}[1]{#1}

% if we're tex4ht'ing, redefine the HTML commands to use tex4ht commands
\ifhtml
  \renewcommand{\htcode}[1]{\HCode{#1}}
  \renewcommand{\htlatex}[1]{}
\else
  \renewcommand{\htlinkhtml}[2]{}
  \renewcommand{\htlinkplainhtml}[1]{}
  \renewcommand{\htlatexhtml}[1]{}
\fi


%%%%%%%%%%%%%%%%%%%%%%%%%%%%%%%%%%%%%%%%%%%%%%%%%%%%%%%%%%%%%%%%%%%%%%%%%%%%
% notes commands 
%%%%%%%%%%%%%%%%%%%%%%%%%%%%%%%%%%%%%%%%%%%%%%%%%%%%%%%%%%%%%%%%%%%%%%%%%%%%

% writes a small "note:" marker on the left margin of the note 
\newcommand{\notes}[1]
{
\setlength{\unitlength}{1mm}
\begin{picture}(0,0)(15, 0)
\put(0,0) {\framebox{{\scriptsize {\sf {\bf note:}}}}}
\end{picture}%
{\sf{#1}}
\begin{picture}(0,0)
\put(0,0){\rule{.5\baselineskip}{.5\baselineskip}}
\end{picture}%
}

% uncomment for final version.
%%% \renewcommand{\notes}[1]{}

% IF pdf mode is enabled it will create a PDF note (non printable)
\newcommand{\pdfnote}[1]{
  \ifpdf
    \pdfannot         % generic annotation
                      
      width 10cm      % the dimension of the annotation can be controlled
      height 0cm      % via <rule spec>; if some of dimensions in
      depth  4cm      % <rule spec> is not given, the corresponding
                      % value of the parent box will be used.
    {                 
      /Subtype /Text  % text annotation
      %/Open true     % if given the text annotation opened by default
      /Contents       % text contents
        (#1)
    }
  \else
  \notes{#1}
  \fi
}


%%%%%%%%%%%%%%%%%%%%%%%%%%%%%%%%%%%%%%%%%%%%%%%%%%%%%%%%%%%%%%%%%%%%%%%%%%%%
% switches 
%%%%%%%%%%%%%%%%%%%%%%%%%%%%%%%%%%%%%%%%%%%%%%%%%%%%%%%%%%%%%%%%%%%%%%%%%%%%

\textwidth 5in
\textheight 8.5in

%\oddsidemargin 0.1in
%\evensidemargin 0.1in
%\topskip 0.35in
%\footskip 0.2in
%\topmargin -1.1cm
\tolerance=6000

% paragraph break with no indent
\setlength{\parindent}{0pt}

% extra spacing between paragraphs
\setlength{\parskip}{2ex}
%\setlength{\parskip}{1ex plus 0.5ex minus 0.2ex}

% number subsubsections in the TOC
\setcounter{tocdepth}{3}

% spacing
\singlespacing
%\onehalfspacing
%\doublespacing



%%%%%%%%%%%%%%%%%%%%%%%%%%%%%%%%%%%%%%%%%%%%%%%%%%%%%%%%%%%%%%%%%%%%%%%%%%%%
% misc commands 
%%%%%%%%%%%%%%%%%%%%%%%%%%%%%%%%%%%%%%%%%%%%%%%%%%%%%%%%%%%%%%%%%%%%%%%%%%%%

\newcommand{\delete}[1]{}
\newcommand{\omitted}[1]{{\small \sf =OMITTED=}}
\newcommand{\thing}{book}

\definecolor{cmdcol}{RGB}{0, 0, 128}
\definecolor{mimirblue}{RGB}{0, 97, 147}
\newcommand{\cmd}[1]{\textcolor{cmdcol}{\textbf{\texttt{#1}}}}

%%%%%%%%%%%%%%%%%%%%%%%%%%%%%%%%%%%%%%%%%%%%%%%%%%%%%%%%%%%%%%%%%%%%%%%%%%%%
% sectioning commands 
%%%%%%%%%%%%%%%%%%%%%%%%%%%%%%%%%%%%%%%%%%%%%%%%%%%%%%%%%%%%%%%%%%%%%%%%%%%%

\newcommand{\chapt}[1]{\section{#1}}
\newcommand{\sect}[1]{\subsection{#1}}
\newcommand{\subsect}[1]{\subsubsection{#1}}
\newcommand{\subsubsect}[1]{{\bf #1}}
\newcommand{\subsubsubsect}[1]{{\bf #1}}
\newcommand{\chapthing}{section}
\newcommand{\Chapthing}{Section}
\newcommand{\chapthings}{sections}
\newcommand{\Chapthings}{Sections}

\renewcommand{\chapt}[1]{\chapter{#1}}
\renewcommand{\sect}[1]{\section{#1}}
\renewcommand{\subsect}[1]{\subsection{#1}}
\renewcommand{\subsubsect}[1]{\subsubsection{#1}}

\renewcommand{\chapthing}{chapter}
\renewcommand{\Chapthing}{Chapter}
\renewcommand{\chapthings}{chapters}
\renewcommand{\Chapthings}{Chapters}


%%%%%%%%%%%%%%%%%%%%%%%%%%%%%%%%%%%%%%%%%%%%%%%%%%%%%%%%%%%%%%%%%%%%%%%%%%%%
% compact itemize etc. 
%%%%%%%%%%%%%%%%%%%%%%%%%%%%%%%%%%%%%%%%%%%%%%%%%%%%%%%%%%%%%%%%%%%%%%%%%%%%

% store the current value of parskip and allow resetting
% via \resetparskip
\newlength{\oldparskip}
\setlength{\oldparskip}{\parskip} 
\oldparskip=\parskip 
\newcommand{\resetparskip}[0]
{
  \parskip=\oldparskip
}

% \bit, \eit - compact version of itemize
% note that parskip is changed. To reset it use \resetparskip
% (after any text following the list that's not a new paragraph)
\newcommand{\bit}[0]
{
  \parskip=-4pt
  \begin{itemize}
  \itemsep=-3pt
}
\newcommand{\eit}[0]
{
  \end{itemize}
  \resetparskip
}

% \ben, \een - compact version of enumerate
% note that parskip is changed. To reset it use \resetparskip
% (after any text following the list that's not a new paragraph)
\newcommand{\ben}[0]
{
  \parskip=-4pt
  \begin{enumerate}
  \itemsep=-3pt
}
\newcommand{\een}[0]
{
  \end{enumerate}
  \resetparskip
}

% \bde, \ede - compact version of description
% note that parskip is changed. To reset it use \resetparskip
% (after any text following the list that's not a new paragraph)
\newcommand{\bde}[0]
{
  \parskip=-4pt
  \begin{description}
  \itemsep=-3pt
}
\newcommand{\ede}[0]
{
  \end{description}
  \resetparskip
}


%%%%%%%%%%%%%%%%%%%%%%%%%%%%%%%%%%%%%%%%%%%%%%%%%%%%%%%%%%%%%%%%%%%%%%%%%%%%
% font size
%%%%%%%%%%%%%%%%%%%%%%%%%%%%%%%%%%%%%%%%%%%%%%%%%%%%%%%%%%%%%%%%%%%%%%%%%%%%

% font size - uncomment for standard print
\newcommand{\nsmall}{\small}
\newcommand{\nnormalsize}{\normalsize}
\newcommand{\nlarge}{\large}

% font size - uncomment for large print
%\newcommand{\nsmall}{\normalsize}
%\newcommand{\nnormalsize}{\Large}
%\newcommand{\nlarge}{\Large}


%%%%%%%%%%%%%%%%%%%%%%%%%%%%%%%%%%%%%%%%%%%%%%%%%%%%%%%%%%%%%%%%%%%%%%%%%%%%
% quotex - including quotes in documents
%%%%%%%%%%%%%%%%%%%%%%%%%%%%%%%%%%%%%%%%%%%%%%%%%%%%%%%%%%%%%%%%%%%%%%%%%%%%

% create the .qt aux file
\newcommand{\qtinit}{
  \newwrite\qtaux
  \immediate\openout\qtaux=\jobname.qta
  \immediate\write\qtaux{! \jobname.qta}
}

% record the name of a quotable file
\newcommand{\qtpath}[1]{
  \immediate\write\qtaux{qtpath: #1}
}

% include a quote
\newcommand{\qt}[1]{
  \newread\qtfile
  \immediate\openin\qtfile=qt_#1.tex
  \ifeof\qtfile
    \immediate\closein\qtfile
    \immediate\write\qtaux{quote: #1}
  \else
    \immediate\closein\qtfile
    \input{qt_#1}
  \fi
}

% commands re-defined by the quotes files
\newcommand{\qtauthor}{}
\newcommand{\qttitle}{}
\newcommand{\qtdate}{}
\newcommand{\qtquote}{}
\newcommand{\qtpage}{}
\newcommand{\qtall}{}


%%%%%%%%%%%%%%%%%%%%%%%%%%%%%%%%%%%%%%%%%%%%%%%%%%%%%%%%%%%%%%%%%%%%%%%%%%%%
%%%%%%%%%%% Support for including code listings %%%%%%%%%%%%%%%%%%%%%%%%%%%%
%%%%%%%%%%%%%%%%%%%%%%%%%%%%%%%%%%%%%%%%%%%%%%%%%%%%%%%%%%%%%%%%%%%%%%%%%%%%

% Generic settings

\lstset{
  language=Java,
  keywordstyle=\color{blue}\bfseries,
  commentstyle=\color[rgb]{0,0.5,0}\it,
  stringstyle=\color[rgb]{1,0,1},
  showstringspaces=false,
  rulesepcolor=\color[rgb]{0.8,0.8,0.8},
  captionpos=b
}

% Normal size for listings
\newcommand{\lstnormal}{
\lstset{
  basicstyle=\footnotesize\ttfamily,
  numbers=left, 
  numberstyle=\tiny,
  stepnumber=1,
  numbersep=10pt,
  xleftmargin=20pt,
  xrightmargin=5mm,
  frame=shadowbox,
  frameround=ffff,
  framesep=6pt,
  framexleftmargin=5mm
}
}

% Define a compact listings mode
\newcommand{\lstcompact}{
\lstset{
  basicstyle=\scriptsize\ttfamily,
  numbers=none, 
  xleftmargin=0pt,
  frame=none,
  breaklines=true
}
}

% by default, we start with the normal size
\lstnormal
 
