\Mimir\ is a multi-paradigm information management index and repository which
can be used to index and search over text, annotations, semantic schemas
(ontologies), and semantic meta-data (instance data). It allows queries that
arbitrarily mix full-text, structural, linguistic and semantic queries and that
can scale to gigabytes of text.

A typical semantic annotation project deals with large quantities of data of
different kinds. \Mimir\ provides a framework for implementing indexing and
search functionality across all these data types, listed below in the order of
increasing information density:

{\bf Text}\\
All documents have a textual content. Support for full text search represents
the most basic indexing functionality and it is required in most (if not all)
cases.  Even when semantic annotation is used to abstract away from the
actual textual data, the original content still needs to be accessible so
that it can be used to provide textual query fragments in the case of more
complex conceptual queries.

\Mimir\ uses inverted indexes\footnote{{\em Inverted Indexes} are data
structures traditionally used in Information Retrieval to support indexing of
text.} for indexing the document content (including additional linguistic
information, such as part-of-speech or morphological roots), and for
associating instance of annotations with the position in the input text where
they occur. The inverted index implementation used by \Mimir\ is based on
MG4J\footnote{\url{http://mg4j.dsi.unimi.it/}}.

{\bf Annotations}\\
The first step in abstracting away from the plain text content is the
production of {\em annotations}. Annotations are meta-data associated to text
snippets in the documents. \Mimir's view of annotations is based on that of
GATE, with each annotation described by
\begin{itemize}
  \item the document it belongs to;
  \item the start and end offset of the referred text snippet;
  \item the annotation type;
  \item an arbitrary set of \verb!<!feature,value\verb!>! pairs.
\end{itemize}

An annotation index supports a more generic search paradigm. Depending on the
type of annotations available, the user can search across different dimensions.
If, for example, the documents are annotated with occurrences of {\tt Person,
Location, Organization} entities, then searches like {\tt \{Person\}, CEO of
\{Organization\}, based in \{Location\}} become possible.  Storage of
annotation data in \Mimir\ indexes is handled by plugins, \Mimir\ ships with
two storage plugins by default, one storing annotation data in a relational
database and the other in a Knowledge Base to support richer semantic querying.

ANNIC (ANNotations In
Context)\footnote{See \url{http://gate.ac.uk/userguide/chap:annic}.} is a tool
predating \Mimir\ that supports the indexing of annotations, and that has been
used to inform the design of \Mimir.

{\bf Knowledge Base Data}\\
Knowledge Base (KB) Data  consists of an ontology populated with instances. The
ontology represents the data schema and comprises a hierarchy of class types
and a hierarchy of properties that are applicable between instances of classes.
The instance data represents facts that are known to the systems and is
typically at least partially derived from semantic annotation over documents.
KB data is used to reach a higher level of abstraction over the information in
the documents which enables conceptual queries such as ``find all mentions of
{\tt Person}s who are employed by any organization based in Yorkshire''.

A KB that is pre-populated with appropriate world knowledge can perform other
generalisations that are natural to humans users, such as being able to identify
Vienna as a valid answer to queries relating to Austria, Europe or the Western
Hemisphere.

As mentioned above, \Mimir\ can make use of a Knowledge Base to store
information relating to annotations. The links between annotations, the textual
data, and the knowledge base information are created by the inclusion into the
text indexes of a set specially-created URIs that are associated with
annotation data. Furthermore, URIs of entities from the Knowledge Base can be
stored as annotation features.

Knowledge bases are typically represented as a collection of triples that are
kept in highly-specialised and optimised triple stores, using standards such as
RDF or one the versions of OWL\footnote{See
\url{http://www.w3.org/RDF/} and \url{http://www.w3.org/TR/owl-features/}.}. The
implementation used by \Mimir\ is based on ORDI and
OWLIM\footnote{See
\url{http://www.ontotext.com/ordi/} and \url{http://www.ontotext.com/owlim/}.}.
