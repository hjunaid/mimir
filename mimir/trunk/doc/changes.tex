This appendix details the main changes in each \Mimir\ release.

\section{Version 4.1.3 (September 2012)}
\begin{itemize}
  \item Bug fix in ranking query runner (used to search local indexes): when
  obtaining metadata fields a document ID was used instead of a document rank.
\end{itemize}
\section{Version 4.1.2 (August 2012)}
\begin{itemize}
  \item Small bug fix to perform more input validation when using a federated 
  index: calling methods with an invalid {\tt rank} parameter will now cause an 
  exception. 
\end{itemize}
\section{Version 4.1.1 (May 2012)}
\begin{itemize}
\item It is now possible to specify an index ID for a newly created/imported
  local, remote or federated index, rather than having to create the index with
  a random UUID and then change the ID later.
%TODO fix up the screenshots to match this  
\item Bugfix: stopped the web search UI from showing `{\tt null}' for context
  tokens outside of the document, when a hit result occurs close to the end of
  the document.
\item Bugfix: the annotation type needed to be specified twice in the index
  template when using the SPQARQL plugin.
\item Bugfix: the web search UI was not updating correctly when a query
  completed without matching any results.   
\end{itemize}

\section{Version 4.1 (May 2012)}
\begin{itemize}
\item A bugfix was applied to avoid leaking threads and memory in the new
  ranking query runner implementation (the class {\tt gate.mimir.search.RankingQueryRunnerImpl}).
\item \Mimir{} now uses the mg4j-big variant of the MG4J library. This uses
  64 bit integers (Java longs) for document identifiers, and allows for larger
  indexes to be created.
\item The dependency to MG4J and related libraries is now managed through the
  maven-central repository.
\end{itemize}

\section{Version 4.0 (February 2012)}
\begin{itemize}
  \item Changed the results presentation to be document-centric, as opposed to
  hit-centric.
  \item Overhauled the query API (in all modalities: Java local, Java remote,
  and XML remote) to work in document centric mode and to remove the main pain
  points identified.
  \item Simplified all the query APIs by making them almost completely
  synchronous.
  \item Added support for ranking the results (see
  Sections~\ref{sec:search:rank}  and \ref{sec:extend:scorers}).
  \item New implementations for all the query runners (used when searching
  local, remote and federated indexes).
  \item Replaced the old GWT based UI with a new implementation (see
  Section~\ref{sec:search:gus}).
  \item Added the mimir-cloud web application to the source tree (see
  Section~\ref{sec:webapps}).
\end{itemize}

\section{Version 3.4.0 (November 2011)}

\begin{itemize}
\item Added support for indexing document metadata, i.e. features (see
Section~\ref{sec:indexing:templates}).
\item \Mimir{} Grails Plugin: moved some configuration options from the external
file to a database field, so that it can now be changed using the admin web UI.
\item API: simplified the construction of all default Semantic Annotation
Helpers. They all get a single no-argument constructor, and set of setter
method for editing the various properties (Java Bean style). The Groovy
interface does not change, as Groovy will automatically convert a constructor
call that takes a Map to a call for the no-argument constructor, followed by all
the required setPropertyXYZ calls.
\item Completely removed the (previously deprecated) {\tt ordi} plugin, as it
relies on software that is no longer supported by the original authors.
\item Removed the {\tt mimir-demo} example application from the source tree. It
can now be automatically generated using an Ant call (see 
Section~\ref{sec:building}).
\item Licence changed to LGPL.
 
\end{itemize}

\section{Version 3.3.0 (October 2011)}

\begin{itemize}
\item Added support for marking documents as ``deleted'' (see
section~\ref{sec:admin:takedown}).

\item Major changes to the format of the Index Template Groovy DSL (see
section~\ref{sec:indexing:templates}).  The old format provided by \Mimir\
3.2.0 is still supported for existing semantic annotation helper types, but
new helper types in future may not be supported in the old style DSL.

\item Added the {\em SPARQL} semantic annotation helper (see
section~\ref{sec:plugins:sparql}).

\item Updated versions of a number of libraries (H2 database to 1.3.160, OWLIM
to 3.5, MG4J to 4.0, fastutil to 6.4, dsiutils to 2.0).

\item The \verb|ordi| semantic annotation helper plugin is now deprecated.  Use
the \verb|sesame| plugin instead, which supports the same on-disk format for
its annotation storage but uses a different library to access it.

\item Fixed various bugs and memory leaks (see subversion logs for full
details).

\end{itemize}

\section{Version 3.2.0 (May 2011)}

First public release of \Mimir, under an AGPL licence.

% vim:ft=tex:
