This appendix details the main changes in each \Mimir\ release.

\section{Version 5.3 (January 2017)}
\begin{itemize}
\item Now depends on GATE Embedded 8.3.
\item Added a tool to repair failed indexes, see section~\ref{sec:tools:repair}
  for details.
\item Added the ability to completely ignore annotations that have none of the
  features configured in the index template.  This is more useful for
  document-mode helpers rather than normal annotation-mode ones, see
  section~\ref{sec:plugins:db} for an explanation.
\end{itemize}

\section{Version 5.2 (June 2016)}
\begin{itemize}
\item Now depends on GATE Embedded 8.2.
\item The Grails plugin and \verb!mimir-cloud! application have been upgraded
  to Grails 2.5.4 for better compatibility with Java 8.  Note that the plugin
  and app will \emph{not} work with Grails 3.
\item The query language can now handle non-ASCII characters in unquoted
  strings.
\item The \Mimir{} indexing PR is now much more efficient, able to send more
  than one document in each HTTP call.
\end{itemize}

\section{Version 5.1 (June 2015)}
\begin{itemize}
\item \Mimir{} 5.1 depends on GATE Embedded 8.1.
\item Bug fixes in various corner cases, in particular for very sparse semantic
  annotations (where annotations of a particular type are found in relatively
  few documents).
\item Robustness improvements in \verb!mimir-client! indexing code.
\item The SPARQL semantic annotation helper can now send queries to the SPARQL
  endpoint using POST requests instead of GET, and now works correctly with
  endpoints that require HTTP basic authentication.
\end{itemize}

\section{Version 5.0.1 (October 2014)}
Two critical fixes:
\begin{itemize}
  \item Deletion of documents now works correctly, it had been broken in
  version 5.0
  \item Fixed clustering logic for multi-batch indexes.
\end{itemize}

\section{Version 5.0 (February 2014)}
\begin{itemize}
  \item \Mimir{} indexes are now updateable: new documents can be submitted for
  indexing at any time.
  \item \Mimir{} indexes are now live: they can index new documents and serve
  queries at the same time. Manually {\em closing} indexes before they become
  searcheable is no longer required.
  \item The {\em mimir-demo} example web application has been removed.
  \item The {\em mimir-cloud} has been modified to make it more suitable as a
  generic example web application.
  \item The sesame \Mimir{} plugin has been removed. For standard annotation
  indexing we recommend using the db-h2 plugin. For handling formal semantics,
  we recommend using the SPARQL plugin.
  \item New query operator: {\bf MINUS} (also `-') performs the set minus
  operation on result sets (see Section~\ref{sec:minus-query}).  
  \item \Mimir{} now supports the construction of direct indexes (see
  Section~\ref{sec:direct-indexes}). Direct indexes are used to support a new
  family of queries, that use document ID as query terms, and which return terms
  as results. Currently these are only available as a Java API, and can be found
  in the {\tt gate.mimir.search.terms} package.
  \item Semantic annotation helpers are now capable of 'describing' a matched
  mention. The S-A-H implementations included in the main distribution provide
  default implementations for this functionality, which can be replaced by
  pluggin-in alternative versions.
  \item The on-disk format for \Mimir{} indexes has changed. This was required
  in order to support live indexing  and searching.
  \item \Mimir{} has been upgraded to use MG4J version 5.2.1. Newly created
  indexes will now be semi-succint, which is the highest performance
  implementation.
  \item \Mimir{} now uses Grails 2.2.3 and GWT 2.6.0 to build the mimir-cloud
  web application.
  \item Bugfix: you can now use a string on the right hand side of a \verb!<!,
  \verb!>!, \verb!<=! and \verb!>=! in annotation queries. This was always
  documented, but did not work before.
  \item Many other bugfixes.
\end{itemize}
\section{Version 4.1.3 (September 2012)}
\begin{itemize}
  \item Bug fix in ranking query runner (used to search local indexes): a 
  document ID was used instead of a document rank when requesting metadata 
  fields.
\end{itemize}
\section{Version 4.1.2 (August 2012)}
\begin{itemize}
  \item Bug fix to void null pointer exceptions when the API is used to access
  query results in a federated index without first checking the number of
  available documents. Calling methods with an invalid {\tt rank} parameter will
  now cause an index out of bounds exception.
\end{itemize}
\section{Version 4.1.1 (May 2012)}
\begin{itemize}
\item It is now possible to specify an index ID for a newly created/imported
  local, remote or federated index, rather than having to create the index with
  a random UUID and then change the ID later.
%TODO fix up the screenshots to match this  
\item Bugfix: stopped the web search UI from showing `{\tt null}' for context
  tokens outside of the document, when a hit result occurs close to the end of
  the document.
\item Bugfix: the annotation type needed to be specified twice in the index
  template when using the SPQARQL plugin.
\item Bugfix: the web search UI was not updating correctly when a query
  completed without matching any results.   
\end{itemize}

\section{Version 4.1 (May 2012)}
\begin{itemize}
\item A bugfix was applied to avoid leaking threads and memory in the new
  ranking query runner implementation (the class {\tt gate.mimir.search.RankingQueryRunnerImpl}).
\item \Mimir{} now uses the mg4j-big variant of the MG4J library. This uses
  64 bit integers (Java longs) for document identifiers, and allows for larger
  indexes to be created.
\item The dependency to MG4J and related libraries is now managed through the
  maven-central repository.
\end{itemize}

\section{Version 4.0 (February 2012)}
\begin{itemize}
  \item Changed the results presentation to be document-centric, as opposed to
  hit-centric.
  \item Overhauled the query API (in all modalities: Java local, Java remote,
  and XML remote) to work in document centric mode and to remove the main pain
  points identified.
  \item Simplified all the query APIs by making them almost completely
  synchronous.
  \item Added support for ranking the results (see
  Sections~\ref{sec:search:rank}  and \ref{sec:extend:scorers}).
  \item New implementations for all the query runners (used when searching
  local, remote and federated indexes).
  \item Replaced the old GWT based UI with a new implementation (see
  Section~\ref{sec:search:gus}).
  \item Added the mimir-cloud web application to the source tree (see
  Section~\ref{sec:mimir-cloud}).
\end{itemize}

\section{Version 3.4.0 (November 2011)}

\begin{itemize}
\item Added support for indexing document metadata, i.e. features (see
Section~\ref{sec:indexing:templates}).
\item \Mimir{} Grails Plugin: moved some configuration options from the external
file to a database field, so that it can now be changed using the admin web UI.
\item API: simplified the construction of all default Semantic Annotation
Helpers. They all get a single no-argument constructor, and set of setter
method for editing the various properties (Java Bean style). The Groovy
interface does not change, as Groovy will automatically convert a constructor
call that takes a Map to a call for the no-argument constructor, followed by all
the required setPropertyXYZ calls.
\item Completely removed the (previously deprecated) {\tt ordi} plugin, as it
relies on software that is no longer supported by the original authors.
\item Removed the {\tt mimir-demo} example application from the source tree. It
can now be automatically generated using an Ant call (see 
Section~\ref{sec:building}).
\item Licence changed to LGPL.
 
\end{itemize}

\section{Version 3.3.0 (October 2011)}

\begin{itemize}
\item Added support for marking documents as ``deleted'' (see
section~\ref{sec:admin:takedown}).

\item Major changes to the format of the Index Template Groovy DSL (see
section~\ref{sec:indexing:templates}).  The old format provided by \Mimir\
3.2.0 is still supported for existing semantic annotation helper types, but
new helper types in future may not be supported in the old style DSL.

\item Added the {\em SPARQL} semantic annotation helper (see
section~\ref{sec:plugins:sparql}).

\item Updated versions of a number of libraries (H2 database to 1.3.160, OWLIM
to 3.5, MG4J to 4.0, fastutil to 6.4, dsiutils to 2.0).

\item The \verb|ordi| semantic annotation helper plugin is now deprecated.  Use
the \verb|sesame| plugin instead, which supports the same on-disk format for
its annotation storage but uses a different library to access it.

\item Fixed various bugs and memory leaks (see subversion logs for full
details).

\end{itemize}

\section{Version 3.2.0 (May 2011)}

First public release of \Mimir, under an AGPL licence.

% vim:ft=tex:
