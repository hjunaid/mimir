\Mimir\ uses the GATE Embedded {\em CREOLE plugin} mechanism to load semantic
annotation helper classes.  A number of plugins are supplied by default with
the \Mimir\ distribution, and those plugins are described in this chapter.
Information on how to create new plugins to provide user-defined helper classes
can be found in section~\ref{sec:extend:helpers}.

\section{The {\tt db-h2} Plugin}\label{sec:plugins:db}

The {\tt db-h2} plugin a plugin that provides a {\em generic} semantic
annotation helper implementation that can be configured for any annotation type
with any features.  The helper provided by {\tt db-h2} uses an embedded
relational database engine (\url{http://www.h2database.com/}) to store the
annotation data, and generally provides the best performance of the standard
generic helpers.

\lstinline!gate.mimir.db.DBSemanticAnnotationHelper! is the helper class
provided by the {\tt db-h2} plugin.  It has a constructor that takes a
{\tt Map} of configuration parameters, and Groovy provides special ``named
argument'' support for {\tt Map}-valued method and constructor parameters,
allowing the following idiom in the index template DSL:
\begin{lstlisting}[texcl, breaklines, breakindent=150pt]
// note the ``import X as Y'', which is another Groovy feature to create an
// alias for an imported class name
import gate.mimir.db.DBSemanticAnnotationHelper as DefaultHelper
// \ldots
semanticAnnotations = {
  index {
    annotation helper:new DefaultHelper(annType:'Person',
    nominalFeatures:["gender", "title"], stringFeatures:["name"])) }
}
\end{lstlisting}

The supported constructor arguments are:
\bde
\item[annType:] the annotation type which the helper is to process.
\item[nominalFeatures:] the names of the features to be indexed that have
  nominal values. An annotation feature is said be nominal if the range of
  possible values is clearly defined and limited in size. There is no hard rule
  regarding the size of the set of permitted values, but, for optimal results,
  this should not exceed a few tens of values.
\item[integerFeatures:] the names of the features to be indexed that have
  integer values (i.e. values that can be converted to a Java {\tt long}
  value).
\item[floatFeatures:] the names of the features to be indexed that have
  floating-point numeric values (i.e. values that can be converted to a Java
  {\tt double} value).
\item[textFeatures:] the names of the features to be indexed that have
  arbitrary text values (as opposed to the nominal case of a fixed list of
  possible values).
\item[uriFeatures:] the names of the features to be indexed that have
  URIs as values.
\ede

The DB-based helper does not distinguish between text- and URI-valued features,
indexing both types in the same way, but it accepts both kinds as arguments.

\section{The {\tt measurements} Plugin}\label{sec:plugins:measurements}

The GATE {\tt Tagger\_Measurements} plugin, introduced in GATE 6.1, is able to
recognise many different kinds of measurement expressions in text.  It
normalises the value and unit of each measurement into the SI system of
measurements and stores these values as features of the Measurement annotation.
For example, the text ``45 cm'' would be annotated with a normalised unit of
metres and a normalised value of 0.45, the text ``18 in'' would also be
normalised to metres, in this case with a normalised value of 0.4572.

The \Mimir\ {\tt measurements} plugin provides a SAH that implements the same
normalisation on queries.  It processes queries for a ``synthetic'' feature
called ``spec'' which represents a measurement specification in a controlled
language and converts constraints on this feature into the corresponding
constraints on the real normalised value and unit features that have been
indexed.  For example, a search for \{Measurement spec="1 to 3 feet"\} would be
treated as a query for measurements whose normalised unit is metres and whose
normalised value is between 0.3048 and 0.9144, which would match both the ``45
cm'' and ``18 in'' examples above.

\subsection{Configuring the Measurements SAH}

To use the measurements helper you need to first ensure that the
{\tt measurements} plugin is loaded into your \Mimir\ instance, then create an
index template that specifies an instance of the helper:
\begin{lstlisting}[texcl]
import gate.mimir.measurements.MeasurementAnnotationHelper

// \ldots
semanticAnnotations = {
  index {
    // Measurement helper with default settings
    annotation helper:new MeasurementAnnotationHelper(
          delegateHelperType:DefaultHelper)
  }
}
\end{lstlisting}

Note that the measurement helper does not need any ``annType'' or
``{\em xxx}Features'' parameters, as it is hard-coded to work only for
annotations that are produced by the measurement tagger PR.  However the
constructor does take a {\tt Map} with other named arguments:
\begin{lstlisting}[firstnumber=6,texcl]
    // Example of how to configure a custom ``units'' file
    annotation helper:new MeasurementAnnotationHelper(
          delegateHelperType:DefaultHelper,
          unitsFile:'resources/americanUnits.dat',
          locale:'en_US')
\end{lstlisting}

The following parameters are supported:
\bde
\item[delegateHelperType (required)] a {\tt Class} object representing the type
  of generic helper that the Measurements helper should delegate to.  This
  class must provide a 6-argument constructor taking the annotation type (a
  String) and five String arrays for the nominal, integer, float, text and URI
  feature names respectively.
\item[unitsFile] the location of the {\tt units.dat} file used to configure the
  measurements parser.  If not specified, a default file provided with the
  {\tt measurements} plugin is used.  This value can be an absolute URL
  (file:/path/to/units.dat) or a relative path which will be resolved against
  the {\tt measurements} plugin directory.
\item[commonWords] the location of the common words file used by the
  measurements parser.  As with the {\tt unitsFile} parameter, if omitted a
  default file bundled with the plugin is used.
\item[locale] the locale under which the measurements will be parsed.  Defaults
  to ``en\_GB'' if unspecified.
\item[encoding] the character encoding used to read the configuration files.
  Defaults to ``UTF-8'' if unspecified.
\item[annType] the annotation type, if something other than the default of
  ``Measurement''
\ede

The measurements SAH is pre-configured with the feature names that the
measurements tagger produces, and attempting to specify any feature name
parameters such as nominalFeatures will cause an error.

\subsubsection{Measurements helper implementation}

The {\tt MeasurementAnnotationHelper} extends the
{\tt DelegatingSemanticAnnotationHelper} base class described above.  It does
not add any behaviour at indexing time, simply passing all the annotations
through directly to its delegate.  However it overrides the {\tt getMentions}
search method to support the ``spec'' feature.

When a query including a spec feature constraint is received, the helper parses
this spec using the measurements parser to obtain a normalised unit and value
or values for the measurement sought.  It then constructs a number of new
constraint sets that match annotations compatible with the spec and then for
each of these alternatives, runs these constraints in combination with the
other non-spec constraints of the original query against the delegate helper.
The final set of URIs returned is the union of the results obtained from the
delegate for all the alternative reformulations of the spec constraint.

As well as being useful in its own right for Measurement annotations, the
measurements helper serves as an example of how to implement your own
special-purpose helper based on the delegating base class.  Feel free to use it
as a template for your own helper implementations.

\section{The {\tt sparql} Plugin}\label{sec:plugins:sparql}

The {\tt sparql} plugin provides a semantic annotation helper that wraps
another helper, adding flexible {\em semantic query} support.  It is intended
to be used with annotations that have one or more URI-valued features whose
values refer to entities in an external knowledge base (accessible at a
standard {\em SPARQL endpoint}).  The SPARQL helper has no effect at indexing
time, simply delegating all calls through to the underlying helper, but at
search time it allows queries for the synthetic feature ``sparql''.  This
feature value is taken to be a SPARQL ``SELECT'' query, which is posted
to the configured SPARQL endpoint.  The variables in the SELECT query must
correspond to the names of features that have been indexed by the underlying
helper, and each row in the result set becomes a standard \Mimir\ query to the
underlying helper.  Any annotations that match any of these new queries will be
treated as a match for the {\tt sparql} constraint.  This process is described
in detail below.

For example, given a helper configured for the public DBPedia endpoint
{\tt http://dbpedia.org/sparql}, the following \Mimir\ query:
\begin{verbatim}
{Person sparql = "SELECT DISTINCT ?inst WHERE {
  ?inst <http://dbpedia.org/ontology/birthPlace>
    <http://dbpedia.org/resource/Sheffield> }"}
\end{verbatim}
would search for all Person annotations that have an ``inst'' feature
containing the URI of an entity in DBpedia that represents a person born in
Sheffield.


\subsection{Creating a SPARQL Helper}

The SPARQL semantic annotation helper class is called
\lstinline!gate.mimir.sparql.SPARQLSemanticAnnotationHelper!.  It has a
{\tt Map} constructor taking the following parameters:

\bde
\item[delegate (required)] the underlying semantic annotation helper that this
  SPARQL helper should wrap.
\item[sparqlEndpoint (required)] the address of the SPARQL endpoint that this
  helper should use when making SPARQL queries.
\item[queryPrefix] an optional prefix that will be prepended to the string
  specified in the {\tt sparql} synthetic feature to form the actual SPARQL
  query that will be sent to the endpoint.  Typically this would be used to
  define appropriate namespace prefixes.
\item[querySuffix] an optional suffix that will be appended to the end of the
  SPARQL queries.  Thus the actual SPARQL query submitted to the endpoint is
  {\em queryPrefix} + {\tt sparql} feature + {\em querySuffix}.
\item[sparqlEndpointUser and sparqlEndpointPassword] username and password used
  to authenticate to the SPARQL endpoint (only HTTP basic authentication is
  supported).  May be omitted if your endpoint does not require authentication.
\ede

The helper also accepts the usual ``annType'' and ``{\em xxx}Features''
parameters but these are not normally required -- if the delegate helper is a
subclass of {\tt AbstractSemanticAnnotationHelper} (which is the case for all
the standard helpers) then the SPARQL helper will take its feature names from
the delegate, so the only time the features need to be specified explicitly for
the SPARQL helper is if the delegate is a custom helper type that does not
extend {\tt AbstractSemanticAnnotationHelper}.

For example, the following configuration would set up a helper for Person
annotations operating against DBpedia, to support the example query above:
\begin{lstlisting}[texcl, breaklines, breakindent=150pt]
import gate.mimir.db.DBSemanticAnnotationHelper as DBH
import gate.mimir.sparql.SPARQLSemanticAnnotationHelper as SPARQLHelper
// \ldots
semanticAnnotations = {
  index {
    annotation helper:new SPARQLHelper(
        sparqlEndpoint:'http://dbpedia.org/sparql',
        delegate:new DBH(annType:"Person", uriFeatures:['inst']))
  }
}
\end{lstlisting}

Alternatively, the helper could be configured with a queryPrefix to set up some
useful namespace prefixes:
\begin{lstlisting}[texcl, breaklines, breakindent=150pt]
semanticAnnotations = {
  index {
    annotation helper:new SPARQLHelper(
      sparqlEndpoint:'http://dbpedia.org/sparql',
      queryPrefix:
       'PREFIX rdfs:<http://www.w3.org/2000/01/rdf-schema#> \
        PREFIX xsd:<http://www.w3.org/2001/XMLSchema#> \
        PREFIX owl:<http://www.w3.org/2002/07/owl#> \
        PREFIX rdf:<http://www.w3.org/1999/02/22-rdf-syntax-ns#> \
        PREFIX dbo:<http://dbpedia.org/ontology/> \
        PREFIX dbr:<http://dbpedia.org/resource/> ',
      delegate:new DBH(annType:"Person", uriFeatures:['inst']))
  }
}
\end{lstlisting}

Note the backslashes which are a Groovy feature to permit a String literal to
be broken across several lines, and also the trailing space before the closing
quotation mark -- the helper simply concatenates the prefix, query and suffix
without any additional space, so required spaces must be part of the prefix or
suffix string itself.  This would allow the example query above to be rewritten
more succinctly as:
\begin{verbatim}
{Person sparql = "SELECT DISTINCT ?inst WHERE {
  ?inst dbo:birthPlace dbr:Sheffield }"}
\end{verbatim}

For annotation types that have only one URI-valued feature it may be desirable
to include the ``\verb!SELECT DISTINCT ?inst WHERE { !'' in the prefix and add
a suffix of ``\verb! }!'', which would reduce the query down to
\begin{verbatim}
{Person sparql = "?inst dbo:birthPlace dbr:Sheffield"}
\end{verbatim}

If your index template includes several ontology-based annotation types sharing
the same SPARQL endpoint and prefixes then listing these in full for each
annotation type will result in a large and unwieldy template.  However, since
the index template is itself a Groovy script it is possible to declare methods
to factor out the common code.  Method declarations must be placed
{\em outside} the {\tt semanticAnnotations} block:
\begin{lstlisting}[texcl, breaklines, breakindent=150pt]
import gate.mimir.db.DBSemanticAnnotationHelper as DBH
import gate.mimir.sparql.SPARQLSemanticAnnotationHelper as SPARQLHelper

def standardHelper(type) {
  return new SPARQLHelper(
      sparqlEndpoint:'http://dbpedia.org/sparql',
      queryPrefix:
       'PREFIX rdfs:<http://www.w3.org/2000/01/rdf-schema#> \
        PREFIX xsd:<http://www.w3.org/2001/XMLSchema#> \
        PREFIX owl:<http://www.w3.org/2002/07/owl#> \
        PREFIX rdf:<http://www.w3.org/1999/02/22-rdf-syntax-ns#> \
        PREFIX dbo:<http://dbpedia.org/ontology/> \
        PREFIX dbr:<http://dbpedia.org/resource/> ',
      delegate:new DBH(annType:type, uriFeatures:['inst']))
}

// \ldots
semanticAnnotations = {
  index {
    annotation helper:standardHelper('Person')
    annotation helper:standardHelper('Location')
    annotation helper:standardHelper('Organization')
  }
}
\end{lstlisting}

\subsection{Format of SPARQL Queries}

This section describes in more detail how the SPARQL queries relate to the
annotations indexed by the underlying semantic annotation helper.  As a simple
example we consider the query for people born in Sheffield:
\begin{verbatim}
{Person sparql = "SELECT DISTINCT ?inst WHERE {
  ?inst dbo:birthPlace dbr:Sheffield }"}
\end{verbatim}
%
The helper will submit the SPARQL query to its configured endpoint, and receive
a response of the form:

\begin{tabular}{|c|}
\hline
{\bf inst} \\
\hline
{\tt http://dbpedia.org/resource/Gordon\_Banks} \\
{\tt http://dbpedia.org/resource/Michael\_Palin} \\
{\tt http://dbpedia.org/resource/David\_Blunkett} \\
\ldots \\
\hline
\end{tabular}

This will then generate a series of queries to the underlying helper of the
form:
\begin{verbatim}
{Person inst = "http://dbpedia.org/resource/Gordon_Banks"}
{Person inst = "http://dbpedia.org/resource/Michael_Palin"}
...
\end{verbatim}
and any annotation that matches any of these queries will be returned as a
match for the {\tt sparql} constraint.

The SPARQL query can bind more than one variable, and the values of the
variable bindings can be RDF literals as well as URIs, they convert to queries
on the underlying helper in the same way.
