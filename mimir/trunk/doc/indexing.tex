\Mimir\ is designed to index semantically annotated documents. It accepts as
input GATE documents\footnote{\url{http://gate.ac.uk/userguide/chap:corpora}}
and produces a set of indexes as a result. The way the text and annotations of
the input documents are converted into indexes is controlled through
configuration options.

\section{Configuring the Indexer}\label{sec:indexing:templates}

In the \Mimir\ web interface, the configuration of a new index is represented
by an {\em index template}.  This specifies:
\begin{itemize}
\item which annotation types and features to index
\item which annotation sets contain these annotations
\item (optionally) which document features should be indexed
\item how to handle the document format and metadata
\end{itemize}

Index templates can be managed using the ``Click here to manage the index
templates'' link at the bottom of the \Mimir\ front page.  An index template is
specified in a structured ``domain specific language'' using Groovy ---
Listing~\ref{lst:default-index-template} shows an example of the default
template provided by the \Mimir\ Grails plugin.
%
% Define a new command for the default index template so we can slice it up
% lower down
%
\newcommand{\defaultIndexTemplate}[1][]{{%
  \lstset{morecomment=[is]{/@}{@/}}%
  \lstinputlisting[breaklines, breakindent=150pt, #1]{default-index-config.txt}%
}}%
%
\defaultIndexTemplate[float=htb,%
    caption={The default index template provided with \Mimir},%
    label=lst:default-index-template,%
]

{\newpage}
The various sections of the template are as follows:

\subsection*{Imports}
\defaultIndexTemplate[linerange=1-5,firstnumber=1]

The template can optionally start with import statements to import any Java
classes that are used further on in the template.

\subsection*{Token Definitions}\label{sec:indexing:tokens}
\defaultIndexTemplate[linerange=7-13,firstnumber=7]

The next section of the template deals with the {\em tokens} that \Mimir\ will
base its index on.  \Mimir\ sees every document as a stream of tokens rather
than a stream of characters, and all the annotations indexed by \Mimir\ are
stored in terms of their starting and ending {\em tokens}, not character
offsets.  Thus for \Mimir\ to work correctly it needs to know how to split up
the document into tokens and what information to store about each token.  For
this purpose it uses GATE annotations, and indexes the values of features on
the annotations.

The following options can be configured:
\begin{description}
\item[tokenASName] The name of the annotation set in which the token
  annotations can be found (for example \lstinline!tokenASName = "mimir"!). To
  use the default annotation set, which has no name, this may be left
  unspecified, or explicitly set to the empty string: \lstinline!""!, or to
  \lstinline!null! (without quotes).
\item[tokenAnnotationType] The annotation type that should be used as tokens.
  This entry is required, and can generally be simply set to the default
  \lstinline!ANNIEConstants.TOKEN_ANNOTATION_TYPE! (with a suitable
  \lstinline!import! at the top of the template).
\item[tokenFeatures] A block of code giving the features from each token
  annotation that should be indexed.
\end{description}

The {\em tokenFeatures} block should list the features to be indexed as shown
in the example, each feature name followed by parentheses.  For advanced users
an MG4J \lstinline!TermProcessor! instance may be provided inside these
parentheses.  By default, if no term processors are specified, the {\em first}
feature is converted to lowercase and the subsequent features are not modified.
Since terms in a query are processed using the same processor as those in the
index, this has the effect of making searches on the first feature
case-insensitive, and searches on the other features case-sensitive. To stop
any processing being done, you should supply a
\lstinline!it.unimi.dsi.mg4j.index.NullTermProcessor! value, by specifying e.g.
\lstinline!string(NullTermProcessor.getInstance())!, after including the
relevant \lstinline!import! statement at the top.

\subsection*{Semantic Annotations}\label{sec:indexing:helpers}
\defaultIndexTemplate[linerange={15-19},firstnumber=15]

The next section defines the {\em semantic annotations} that \Mimir\ will
include in the index.  Each \lstinline!index! block defines a set of semantic
annotation types that will be indexed and stored together in one index.  The
choice of how to group annotation types together into indexes can affect the
indexing speed, as the annotations within one index are processed sequentially
by a single thread, whereas types in separate indexes are indexed in parallel.

Each annotation type to be indexed is introduced by ``{\tt annotation}''.  This
is a method call in the Groovy DSL which takes the following named arguments:
\bde
\item[helper] The {\em semantic annotation helper} Java object that should be
  used to index this annotation type.
\item[type] The annotation type that is to be indexed.  When using the default
  semantic annotation helper types this can be omitted.\footnote{In particular,
  if the specified helper has a method ``{\tt getAnnotationType()}'' then this
  will be called and the returned value used as the annotation type.  All the
  standard helpers provided with \Mimir\ extend
  {\tt AbstractSemanticAnnotationHelper} which implements this method.}
\ede

\subsubsection{Semantic Annotation Helpers}

Semantic annotations are stored in special indexes that associate URIs with
document positions. During indexing, the role of the helper implementations is 
to store the necessary information about each annotation to be indexed in a
persistent form and return one or more URIs that identify it.

One could make a distinction between {\em generic} semantic annotation helper
types, which can be configured to handle any annotation type and features, and
{\em special-purpose} helpers that are designed to handle specific annotation
types.  \Mimir\ supplies a generic helper implementations in the {\tt db-h2}
plugin that store annotation information in a relational database.  For the most
standard cases, this default helper implementation should be sufficient.  One
sample special-purpose helper for {\tt Measurement} annotations (as generated by
the GATE {\tt Tagger\_Measurements} plugin) is also provided, in the {\tt
measurements} plugin.  This is intended both to be useful in its own right and
to serve as a template for how to implement your own helpers for other complex
annotation types.  The {\tt sparql} plugin provides a helper that can wrap any
other helper and add the ability to query for URI-valued features by making a
query to a SPARQL endpoint.

The plugins that include all the provided semantic annotation helpers are
discussed in detail in Chapter~\ref{sec:plugins}. Detailed documentation for
configuring each of the helpers is available there.

{\bf Note for users upgrading from \Mimir\ 3.2.0 and earlier:} the previous
index template DSL style using the annotation type as the method name and the
nominalFeatures etc. as parameters is still supported but should be considered
deprecated.  You should consider porting your index templates to the new style,
as support for the old style may be removed in a future release.

\subsection*{Document Features}\label{sec:indexing:docfeats}
Starting with \Mimir\ version 3.4.0, annotation helpers can also be used to
index document features.
\clearpage
\defaultIndexTemplate[linerange={25-25},firstnumber=25]

The above declaration (note the {\tt mode} parameter!) creates a new Semantic
Annotation Helper that uses the document features instead of the features from
any given annotation. The helper behaves as if a single annotation, of the
declared type (in our case {\it Document}), existed that covers the whole extent
of the document, and has the same features as the GATE Document being indexed.

Things to note:
\begin{itemize}
  \item All helper implementations supplied with \Mimir{} are capable of working
  in {\tt DOCUMENT} mode, so you can use them for indexing document features.
  \item The default value for the {\tt mode} parameter for all supplied helper
  implementations is {\tt ANNOTATION}. Not specifying a {\tt mode} value
  preserves the default functionality (from versions preceding 3.4.0), i.e.
  indexing {\bf annotation} features.
  \item You can have as many helpers as you want working in {\tt DOCUMENT}
  mode, in parallel.
  \item The specified value for the {\tt annType} parameter is used by the
  helper for simulating the presence of an actual annotation spanning the whole
  document; \Mimir\ then behaves as if such annotations actually existed. This
  implies that you cannot re-use the name of an annotation type that is already
  being indexed. For example, if you are already indexing actual annotations of
  type {\tt Document}, then you will need to choose a different name for the
  virtual annotation type used when indexing document features.
\end{itemize}

\subsection*{Document Rendering and Metadata}
\defaultIndexTemplate[linerange={28-30},firstnumber=28]

The final part of the template concerns how document-level metadata is indexed,
and how this can be combined with the document text to render the document
content at search time, with matches of the query highlighted.  These tasks are
performed by objects that implement the interfaces
\lstinline!DocumentMetadataHelper! and \lstinline!DocumentRenderer!
respectively (both in the \lstinline!gate.mimir! package). \Mimir\ provides a
single class \lstinline!gate.mimir.index.OriginalMarkupMetadataHelper!
which implements both interfaces, so in most cases the same object can be used
for both jobs.

An index template must define one \lstinline!documentRenderer! and may define
any number of \lstinline!documentMetadataHelpers! (in a square-bracketed list).
If the renderer is an \lstinline!OriginalMarkupMetadataHelper! (or a subclass)
then the renderer object must be included in the list of metadata helpers in
order to function correctly.  Other metadata helpers may be added to the list
if required.

In the listing above, we use one instance of
\lstinline!OriginalMarkupMetadataHelper! as a document renderer. To enable it to
fucntion, we also include the same object instance in the list of metadata
helpers. Additionally, we also construct an instance of
\lstinline!DocumentFeaturesMetadataHelper!, which we name
\lstinline!documentFeaturesHelper!, and we then add to the list of metadata
helpers. \lstinline!DocumentFeaturesMetadataHelper! instances can be used to
store additional metadata in the index being constructed. The metadata values to
be stored must be provided in the form of GATE document features on the
documents being indexed. When such values are present at indexing time, they are
serialised and stored in the index. At search time, the stored metadata fields
can be retrieved back from the index. Note that the values used must be
serialisable to be usable (i.e. they must implement the 
\lstinline!java.io.Serializable!) interface. 

\subsection*{Direct Indexes}
\label{sec:direct-indexes}
Starting with version $5.0$, \Mimir{} can build direct indexes as well as
inverted ones. By default only inverted indexes are created, which are used to associate
terms to documents. Direct indexes encode the inverse relation from documents to
terms, hence a direct index can be used to find out which terms occur in any
given document.

To enable direct indexes for tokens, the configuration in the index template
needs to be modified like in the following example:
\begin{lstlisting}
tokenFeatures = {
  string(directIndex:true)
  category(directIndex:true)
  root()
}
\end{lstlisting}

In this case, direct indexes will be built for the {\tt string} and {\tt
category} features of \verb!{Token}! annotations, but not for the {\tt root}
feature.

In the case of semantic annotations, direct indexes are enabled in a similar
fashion:
\begin{lstlisting}
index(directIndex:true) {
  annotation helper:new DefaultHelper(...)
  ...
}
\end{lstlisting}

Note that direct indexes can only be enabled at the level of a {\tt index}
element in the template, and not for individual annotation types.

Direct indexes are stored in separate files from the default indirect indexes,
so they will not affect the functionality that does not require direct indexes
at all.

Direct indexes can currently only be searched via the Java API provided by the
{\tt gate.mimir.search.terms} package.

\section{Adding Documents to an Index}\label{sec:indexing:add-docs}

Once an index has been created in {\em indexing} mode, the next stage is to add
documents to the index.  \Mimir\ provides an HTTP API for this which accepts
documents for indexing via HTTP POST requests that include the document in Java
serialised format.  The easiest way to make use of this API is via GCP (the
GATE Cloud Paralleliser batch processing tool) using a
\lstinline!MimirOutputHandler!.  This GCP output handler makes use of the
\lstinline!gate.mimir.index.MimirConnector! (in the {\tt mimir-client} module)
to actually make the remote call, and you can use the same API in your own
code.  To add a GATE document to an open index simply call:
\begin{lstlisting}[breaklines]
MimirConnector.defaultConnector().sendToMimir(document, uri, indexUrl);
\end{lstlisting}
%
\ldots{}with the following parameters:
\bde
\item[document] a \lstinline!gate.Document! for indexing.
\item[uri] the URI that should be used to identify the document in the \Mimir\
  index.  May be \lstinline!null!, in which case \Mimir\ will generate a URI,
  but in most cases there will be a more meaningful identifier that could be
  used.
\item[indexUrl] a \lstinline!java.net.URL! pointing to the location of the
  \Mimir\ index.  This is the ``Index URL'' given on the index information page.
\ede

The document to be indexed must, of course, contain the token and semantic
annotations that the index expects.

It is possible to create your own private instance of
\lstinline!MimirConnector! rather than simply using the default one, but this
is not necessary in normal use.

\section{The Default Representation Scheme}\label{sec:indexing:dsah-detail}

The default generic SAH implementations try to minimise the amount of data
stored in their underlying database or semantic repository by creating
representation templates that are shared between all occurrences of annotations
with the same values for the features. There are two levels of templates, the
first defined by the values of nominal features, and the second that uses the
values of all the other features. This is intended to reflect the typical
scenario where most annotations are defined by a small set of nominal features,
with a few of them having features with arbitrary values. Most annotation types
would then only make use of level-1 templates, with a few of them employing both
level-1, and level-2 templates.

\begin{figure*}[htb]
\begin{center}
{\footnotesize  
\begin{tabular}{|r|l|l|l|l|l|l|l|}
\hline
Document: & London & is & located & on & the & Thames & .\\
\hline
position: & 0 & 1 & 2 & 3 & 4 & 5 & 6\\
\hline
{\bf string:} & london & is & located & on & the & thames & .\\
\hline
{\bf root:} & london & be & locate & on & the & thames & .\\
\hline
{\bf part-of-speech:} & NNP & VBZ & VBN & IN & DT & NNP & .\\
\hline
{\bf Location:} & {\tt type=city} &  &  & &  & {\tt type=river} & \\
\hline
\end{tabular}

\smallskip

\begin{tabular}{ll}
\multicolumn{2}{c}{\bf Token indexes}\\

% Token root index
\parbox[t]{5em} {
\begin{tabular}[t]{|l|l|}
\hline
\multicolumn{2}{|l|}{\bf root index}\\
\hline
. & $0(6)$\\ 
\hline
be & $0(1)$\\ 
\hline
locate & $0(2)$\\
\hline
london & $0(0)$\\
\hline
on & $0(3)$\\
\hline
thames & $0(5)$\\
\hline
the & $0(4)$\\
\hline
\end{tabular}
} &
% Token PoS index
\parbox[t]{6em} {
\begin{tabular}[t]{|l|l|}
\hline
\multicolumn{2}{|l|}{\bf PoS index}\\
\hline
. & $0(6)$\\
\hline
DT & $0(4)$\\ 
\hline
IN & $0(3)$\\ 
\hline
NNP & $0(0, 5)$\\
\hline
VBN & $0(2)$\\
\hline
VBZ & $0(1)$\\
\hline
\end{tabular}
} \\

{\bf Location templates} &  {\bf Location index} \\
% Location templates
\parbox[t]{24em} {
\begin{tabular}[t]{|l|l|}
\hline
{\bf L1 ID} & {\bf type}\\
\hline
1 & city\\ 
\hline
2 & river\\ 
\hline
\end{tabular}

\smallskip

\begin{tabular}[t]{|l|l|l|}
\hline
{\bf L2 ID} & {\bf L1 ID} & {\bf instURI}\\
\hline
1 & 1 & dbpedia.org/resource/London \\ 
\hline
2 & 2 & dbpedia.org/resource/Thames\_river \\ 
\hline
\end{tabular}

\begin{tabular}[t]{|l|l|l|l|}
\hline
{\bf Mention ID} & {\bf L1 ID} & {\bf L2 ID} & {\bf length} \\
\hline
Location:1 & 1 & - & 1 \\
\hline
Location:2 & 1 & 1 & 1 \\ 
\hline
Location:3 & 2 & - & 1 \\
\hline
Location:4 & 2 & 2 & 1 \\
\hline
Location:5 & 2 & 2 & 3 \\ 
\hline
\end{tabular}
} &

% Location index
\parbox[t]{14em} {

\begin{tabular}[t]{|l|l|}
\hline
\multicolumn{2}{|l|}{\bf \{Location\} index}\\
\hline
Location:1 & $0(0)$\\
\hline
Location:2 & $0(0)$\\
\hline
Location:3 & $0(5)$\\ 
\hline
Location:4 & $0(5)$\\ 
\hline
\end{tabular}
}
\end{tabular}
} 
\caption[\Mimir{} index contents]{A very simple example document and the
corresponding contents of a \Mimir{} index. We assume that the only document ID is $0$.\\
Different {\em views} of the document text are generated by different token
features, which are stored in separate sub-indexes. The document string has been
down-cased prior to indexing; we do not show the {\tt string} index, as it is
very similar to the one for the {\tt root} feature. The values used for
Part-of-Speech (PoS) are standard tags as produced by GATE's PoS Tagger:
DT=determiner, IN=preposition, NNP=proper noun, VBN=verb - past participle, 
VBZ=verb - 3rd person singular present.\\
A single annotation type ({\tt \{Location\}}) is being indexed, with two
different occurrences, and we assume the only non-nominal feature to be the
DBpedia instance URI. Note that ``{\tt Location:5}'' (i.e. a mention of the
Thames that is 3-tokens long) does not actually occur in the document text, so
it is not present in the index. We have included it here as an example of an
annotation of length greater than $1$.}
\label{fig:token-indexes}
\end{center}
\end{figure*}

For each input annotation the following IDs are retrieved (or generated on first
occurrence):\\
{\bf Level-1 template ID} The annotation type and the values for all its nominal
features form a tuple. The first time each tuple configuration is seen, it is
allocated a level-1 ID. Subsequent annotations that match an already existing
tuple will re-use the same level-1 ID. For example, in
Figure~\ref{fig:token-indexes} all annotations of type {\em Location} with
feature {\em city} will use the level-1 ID `{\tt 1}'.\\
{\bf Level-2 template ID} The level-1 template ID together with the values for
all the remaining (i.e. non-nominal) features form a second tuple. Unique
configurations of these tuples are allocated level-2 IDs. It should be noted
that most NLP annotations tend to include only nominal features, so they would
not require a level-2 ID. The \verb!{Location}! annotations shown in
Figure~\ref{fig:token-indexes} have a non-nominal feature, so they each get a
level-2 ID allocated to them. Note, however, that all further mentions of e.g.
the {\em Thames} would re-use the same IDs, even when phrased differently in
the text, e.g ``{\em the river Thames}'', or ``{\em La Tamise}''.
\\
{\bf Mention ID} The level-1 ID and the annotation length (number of tokens)
forms a tuple, which is associated with a mention ID -- in
figure~\ref{fig:token-indexes} {\em Location} annotations with feature {\em
city} covering one token will take the mention ID ``Location:1''. If present,
the level-2 ID and the annotation length also get a mention ID. For example, all
mentions of ``the River Thames'' are associated with the mention ID
``{\tt Location:5}'' (because they refer to the Thames, and are 3 tokens long).

Finally, the one or two mention IDs associated with each annotation are added to
an \emph{annotation index}, using the annotation start token as
the position. 

We index two separate mention IDs associated with either level-1 or level-2 IDs,
in order to speed-up searches that only make use of nominal features. For
annotation types that have non-nominal features, the number of level-2 IDs will
be orders of magnitude greater than that for level-1. If a search only relies on
nominal constraints (a large proportion of searches tend to fall into this
category), then the query can be answered much faster by only accessing the
smaller number of posting lists for the matching level-1 IDs.
